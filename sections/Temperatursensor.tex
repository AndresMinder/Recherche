\section{Temperatursensor}
Auf ist die Auswahl für weitere Temperatursensoren gross:\\
\url{http://www.ti.com/de-de/sensors/temperature-sensors/overview.html#} 
\subsection{LM35D}
Die Ausgangsspannung des Temperatursensors LM35D ist linear proportional zur Temperatur in °C. Damit bietet der IC der Produktreihe LM35 mit einer Genauigkeit von $\pm$ 0.5$^{o}$C einen Vorteil gegenüber linearen, in Kelvin kalibrierten Temperatursensoren, da der Benutzer keine hohe Konstantspannung vom Ausgang abziehen muss, um eine praktische $^{o}$C-Skala zu erhalten. Dieser Sensor erfordert weder externes Kalibrieren noch externes Trimmen, um Messungen mit einer Genauigkeit von typ. $\pm$ 1/4$^{o}$C bei Raumtemperatur und $\pm$ 3/4$^{o}$C über den gesamten Temperaturbereich von -55$^{o}$C bis 150$^{o}$C zu gewährleisten. Die niedrige Ausgangsimpedanz, der lineare Ausgang und die präzise inhärente Kalibrierung des Temperatursensors vereinfachen die Verbindung mit Auslese- und Steuerkreisen. Dieser Baustein wird mit einer einfachen Stromversorgung oder mit einer positiven und negativen Stromversorgungseinheit verwendet. Da der Sensor nur 60$\mu$A aufnimmt, beträgt die Eigenerwärmung bei ruhender Luft unter 0.1$^{o}$C.\todo[inline]{Nur bedingt ESD-sicher! Baustein muss bei Lagerung oder Handhabung kurzgeschlossen sein oder in leitfähigem Schaumstoff befinden, um Schäden an den sich MOS-Gates durch elektrostatische Entladung zu verhindern}
% Table generated by Excel2LaTeX from sheet 'Tabelle1'
\begin{table}[htbp]
  \centering
  \caption{Add caption}
    \begin{tabular}{|l|l|r|}
    \toprule
    \rowcolor[rgb]{ .816,  .808,  .808} Anbieter & Supply & \multicolumn{1}{l|}{Preis/stk} \\
    \midrule
    Farnell & 4-30V & 2,17CHF \\
    \bottomrule
    \end{tabular}%
  \label{tab:addlabel}%
\end{table}%
