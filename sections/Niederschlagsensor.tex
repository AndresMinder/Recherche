\section{Niederschlag}
Es gibt sehr viele verschiedene Möglichkeiten den Niederschlag zu messen. Aus Rücksicht zum Projekt 5 und deren Umsetzbarkeit wäre es am schlausten die Niederschlagsmessung (elektrisch mit geloggten Daten) als Wunschziel zu definieren. Ein analoges Messverfahren mit einem Füllbecher zur optischen Betrachtung vor Ort wäre also eine Alternative. Anderenfalls könnten, falls alle Sensordaten auf einer Webpage geloggt werden würden, die Niederschlagsdaten von einer offiziellen Wetterseite bezogen werden. Wichtig zur weiteren Betrachtung sind die benötigten/unterstützen Schnittstellen/Protokolle RS-585 und SDI-12. Diese sind für den Datenaustausch der Loggeinheit und des Sensors wichtig.
\subsection{Regenmelder}
\begin{multicols}{2}
{ \centering
\includegraphics[width=0.4\columnwidth]{graphics/regenmelder.jpg}\\
\captionof{figure}{Regenmelder}
\label{regenmelder}
}
\columnbreak
Es gibt Regenmelder, welche einfach nur den Regen detektieren. Dafür können kapazitive Regensensoren verwendet werden, welche herausfinden ob es regnet oder nicht. Diese müssen seitlich montiert werden, dass das Wasser abläuft und sie müssen zwingend beheizt sein damit der Sensor wieder trocknen kann.
\end{multicols}

\subsection{Niederschlagsmesssensor}
Mit diesem Instrument kann der Niederschlag gemessen werden, welcher in einem bestimmten Zeitintervall gefallen ist (Regen, Schnee wenn dieser zu seinem Wasseräquivalent geschmolzen ist und allenfalls Hagel). Dabei gibt es für dieses Projekt verschiedene Verfahren zur Messung, welche in Frage kommen könnten\footnote{wurden direkt priorisiert}:
\begin{itemize}
\item[1.] Disdrometer
\item[2.] Kippwaage
\item[3.] Wägeprinzip
\end{itemize}
\begin{figure}[hbtp]
\centering
\includegraphics[width=0.85\textwidth]{graphics/vergleich_verfahren.PNG}
\caption{Vergleich der Verfahren}
\label{vergleich_der_verfahren}
\end{figure}

\subsubsection{Disdrometer}
Ein Disdrometer misst im Gegensatz zu den anderen beiden Messverfahren nicht nur die Niederschlagsmenge, sondern auch die Niederschlagsart. Die Messung erfolgt zeit kontinuierlich. Für das Projekt würde also zum Beispiel der \textit{WS100} der Firma Lufft in Frage kommen. Allerdings ist der Preis unbekannt und müsste bei der Firma nachgefragt werden\footnote{Datenblatt der technischen Daten ist im Anhang hinterlegt}.

\subsubsection{Kippwaage}
Das Grundprinzip ist eine Waage mit zwei Behältern im Innern und oben einen trichterförmigen Eingang. Ist ein Behälter gefüllt, dann kippt das System und entleert sich direkt. Über einen Datenlogger wird die Anzahl Kippbewegungen geloggt, wobei das Signal von einem Reed-Kontakt generiert wird.. Das Problem ist, dass dieser Sensor wartungsintensiv ist. Zudem muss wegen Verschmutzungen darauf geachtet werden, wo er positioniert wird.

\subsubsection{Wägeprinzip}
Das Wägeprinzip erfordert einen zu großen zylinderförmigen Behälter. Dieser würde die Wetterstation schwerer und unhandlicher machen.